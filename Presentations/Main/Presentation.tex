% -*- TeX:PL -*-
% $Id: $
\documentclass[12pt]{beamer}
\usepackage[T1]{polski}
\usepackage[cp1250]{inputenc}
\usepackage{lipsum}
\usepackage{multimedia}

\usepackage{mathtools}
\usepackage[ruled,vlined]{algorithm2e}
%\usepackage[a4paper]{geometry}
%\usepackage{graphicx}
%\usepackage[]{hyperref}
%

%
\author{Autor: Tomasz Strza�ka \\ Promotor: dr hab. Pawe� Zieli�ski, prof. PWr}
\title{Wybrane problemy \\ odpornej optymalizacji dyskretnej}
 \subtitle{z mo�liwo�ci� modyfikacji}
 \date{\today}
 \institute{Wydzia� Podstawowych Problem�w Techniki}
 \subject{Logotyp Politechniki Wroc�awskiej}
 \keywords{logotyp, Politechnika Wroc�awska, LaTeX}

\setbeamersize{text margin left=0mm,text margin right=2.5mm}
%\setbeamersize{sidebar width left=0mm,sidebar width right=0cm}
\usetheme[lang=pl,pasek=pasek1]{pwr}

\setbeamercovered{transparent=25}

\newtheorem{pwrblock}{Definicja}
\newenvironment<>{pwrblock}[1][]{%
	\setbeamertemplate{blocks}[rounded][shadow=true]
	\setbeamercolor{block title}{fg=white,bg=red!75!black}%
	\begin{block}#2{#1}}{\end{block}}


\transdissolve[duration=0.2]
\logo{\includegraphics[height=7mm]{poziom-pl-pantone}}
\logo{\includegraphics[height=7.05mm]{logopwrpl}}

\begin{document}
\begin{frame}[plain,t]
 \maketitle
\end{frame}

% % % % % % % % % % % % % % % % % % % % % % % % % % % % % 
% Frame 1
% % % % % % % % % % % % % % % % % % % % % % % % % % % % %

\begin{frame}%[allowframebreaks=.7]
	\frametitle{Plan prezentacji}
	
	\begin{pwrblock}[Plan]
		\begin{itemize}
			\item Problem: \textit{Incremental Minimum Spanning Tree},
			\pause
			\begin{itemize}
				\item idea,
				\pause
				\item \textsc{LP} a \textit{Binary-Based IMST}.
				\pause
			\end{itemize}
			\item Problem: \textit{Recoverable Robust Minimum Spanning Tree},
			\pause
			\begin{itemize}
				\item problem: \textsc{Min-Max},
				\pause
				\item problem adwersarza.
				\pause
			\end{itemize}
			\item Algorytm: \textit{Tabu Search}.
		\end{itemize}
	\end{pwrblock}

\end{frame}

% % % % % % % % % % % % % % % % % % % % % % % % % % % % % 
% Frame 2
% % % % % % % % % % % % % % % % % % % % % % % % % % % % %


\begin{frame}%[allowframebreaks=.7]
	\frametitle{Incremental Minimum Spanning Tree}
	
	\begin{pwrblock}[Definicja]
		\begin{gather*}
			G = \left( V, E \right)\text{,} \quad \left| V \right| = n\text{,} \quad \left| E \right| = m\text{,}\\
			\pause
			\textbf{s} = \left[ c_{1}^{\textbf{s}}, c_{2}^{\textbf{s}}, \dots, c_{m}^{\textbf{s}} \right]\text{,}\\
			\pause
			G_{\textbf{s}} = \left( V, E, \textbf{s} \right)\text{,}\\
			\pause
			\forall i \in \left\{ 1, 2, \dots, m \right\} \quad c_{e_{i}} = c_{i}^{\textbf{s}}\text{.}\\
			\pause\\
			\exists \; \textbf{s}, \textbf{s}^{\prime} \quad : \quad \exists c_{i}^{\textbf{s}} \neq c_{i}^{\textbf{s}^{\prime}}\text{,}\\
			\pause
			T^{\ast}_{\textbf{s}} = \min arg_{T \in \mathcal{T} \left( G_{\textbf{s}} \right)} \sum_{\mathclap{e \in T}} c_{e}\text{.}
		\end{gather*}
	\end{pwrblock}
	
\end{frame}

% % % % % % % % % % % % % % % % % % % % % % % % % % % % % 
% Frame 3
% % % % % % % % % % % % % % % % % % % % % % % % % % % % %


\begin{frame}%[allowframebreaks=.7]
	\frametitle{Incremental Minimum Spanning Tree}
	
	\begin{pwrblock}[Definicja c.d.]
		\begin{gather*}
			T^{\ast}_{\textbf{s}} = \min arg_{T \in \mathcal{T} \left( G \right)} \sum_{\mathclap{e \in T}} c_{e}^{\textbf{s}}\text{,}\\
			\pause
			f : \mathcal{T} \left( G \right) \times \mathcal{T} \left( G \right) \rightarrow \mathbb{Z}^{+}\text{,}\\
			f \left( T, T^{\prime} \right) = \left| T \setminus T^{\prime} \right| = \left| T^{\prime} \setminus T \right|\text{,}\\
			\pause
			\mathcal{T} \left( G, T, k \right) = \left\{ T^{\prime} : T^{\prime} \in \mathcal{T} \left( G \right) \; \wedge \; f \left( T, T^{\prime} \right) \leqslant k \right\}\text{,}\\
			\pause\\
			T^{\ast} = \min arg_{T^{\prime} \in \mathcal{T} \left( G, T, k \right)} \sum_{\mathclap{e \in T^{\prime}}} c_{e}\text{.}
		\end{gather*}
	\end{pwrblock}
	
\end{frame}


% % % % % % % % % % % % % % % % % % % % % % % % % % % % % 
% Frame 4
% % % % % % % % % % % % % % % % % % % % % % % % % % % % %


\begin{frame}%[allowframebreaks=.7]
	\frametitle{Problem Incremental - LP}
	
	\begin{pwrblock}[Model \textsc{LP}]
		\begin{subequations}
			\begin{alignat*}{4}
			& \text{min} & & \sum_{\mathclap{e \in E}} c_{e} \cdot x_{e}\text{,}\\
			& \text{s.t.} & \quad & A \cdot \textbf{x} = \textbf{b}\text{,} & & &\\
			& & & \sum_{\mathclap{e \in T^{\ast} \setminus T^{\ast}_{\textbf{s}}}} x_{e} \leqslant k \text{,} & & &\label{mod:mst7:c}\\
			& & & \phantom{\sum} x_{e} \geqslant 0\text{,} & \forall e \in E\text{,}&&&
			\end{alignat*}
		\end{subequations}
	\end{pwrblock}
\end{frame}


% % % % % % % % % % % % % % % % % % % % % % % % % % % % % 
% Frame 5
% % % % % % % % % % % % % % % % % % % % % % % % % % % % %


\begin{frame}%[allowframebreaks=.7]
	\frametitle{Problem Incremental - relaksacja}
	
	\begin{pwrblock}[Relaksacja ogranicze� modelu]
		\begin{subequations}
			\begin{alignat*}{4}
			& \text{min} & & \sum_{\mathclap{e \in E}} c_{e} \cdot x_{e} + \lambda \cdot \left( \quad \sum_{\mathclap{e \in T^{\ast} \setminus T^{\ast}_{\textbf{s}}}} x_{e} - k \quad \right)\text{,}\\
			& \text{s.t.} & \quad & A \cdot \textbf{x} = \textbf{b}\text{,} & & &\\
			& & & \phantom{\sum} x_{e} \geqslant 0\text{,} & \forall e \in E\text{,}&&&
			\end{alignat*}
		\end{subequations}
	\end{pwrblock}
\end{frame}

% % % % % % % % % % % % % % % % % % % % % % % % % % % % % 
% Frame 6
% % % % % % % % % % % % % % % % % % % % % % % % % % % % %


\begin{frame}%[allowframebreaks=.7]
	\frametitle{Problem Incremental - cel}
	
	\begin{pwrblock}[Definicja c.d.]
		\begin{gather*}
			\sum_{e_{i} \in E \setminus T^{\ast}_{\textbf{s}}} c_{i} \cdot x_{i} + \sum_{\mathclap{e_{i} \in T^{\ast}_{\textbf{s}}}} \left( c_{i} - \lambda \right) \cdot x_{i}\text{,}\\
			\pause\\
			f \left( T^{\ast}, T^{\ast}_{\textbf{s}} \right) \leqslant k \; \wedge \; \lambda = 0 \quad \text{lub}\\
			f \left( T^{\ast}, T^{\ast}_{\textbf{s}} \right) = k \; \wedge \; \lambda \neq 0\text{,}\\
			\pause\\
			\lambda^{\ast} \; : \; T^{\prime} \in \mathcal{T} \left( G_{\textbf{s} \left( \lambda^{\ast} \right)}, T, k \right) \; \wedge \; f \left( T, T^{\prime} \right) = k\text{,}\\
		\end{gather*}
	\end{pwrblock}
\end{frame}


% % % % % % % % % % % % % % % % % % % % % % % % % % % % % 
% Frame 7
% % % % % % % % % % % % % % % % % % % % % % % % % % % % %


\begin{frame}%[allowframebreaks=.7]
	\frametitle{Problem Incremental - przyk�ady}
	
	\begin{pwrblock}[Scenariusz $\textbf{s}$, $k = 1$, $T^{\ast}_{\textbf{s}}$]
		\begin{figure}[htbp]
			\centering
			\pgfimage[width=75mm,interpolate=false]{frame7/a}
		\end{figure}
	\end{pwrblock}
\end{frame}


% % % % % % % % % % % % % % % % % % % % % % % % % % % % % 
% Frame 8
% % % % % % % % % % % % % % % % % % % % % % % % % % % % %


\begin{frame}%[allowframebreaks=.7]
	\frametitle{Problem Incremental - przyk�ady}
	
	\begin{pwrblock}[Scenariusz $\textbf{s}^{\prime}$, $k = 1$, $T^{\ast}_{\textbf{s}^{\prime}}$]
		\begin{figure}[htbp]
			\centering
			\pgfimage[width=75mm,interpolate=false]{frame8/b}
		\end{figure}
	\end{pwrblock}
\end{frame}


% % % % % % % % % % % % % % % % % % % % % % % % % % % % % 
% Frame 9
% % % % % % % % % % % % % % % % % % % % % % % % % % % % %


\begin{frame}%[allowframebreaks=.7]
	\frametitle{Problem Incremental - przyk�ady}
	
	\begin{pwrblock}[Scenariusz $\textbf{s}^{\prime}_{\lambda_{1}}$, $k = 1$, $T \left( \lambda_{1} \right)$]
		\begin{figure}[htbp]
			\centering
			\pgfimage[width=75mm,interpolate=false]{frame9/c1}
		\end{figure}
	\end{pwrblock}
\end{frame}

% % % % % % % % % % % % % % % % % % % % % % % % % % % % % 
% Frame 10
% % % % % % % % % % % % % % % % % % % % % % % % % % % % %


\begin{frame}%[allowframebreaks=.7]
	\frametitle{Problem Incremental - przyk�ady}
	
	\begin{pwrblock}[Scenariusz $\textbf{s}^{\prime}_{\lambda_{2}}$, $k = 1$, $T \left( \lambda_{2} \right)$]
		\begin{figure}[htbp]
			\centering
			\pgfimage[width=75mm,interpolate=false]{frame10/d1}
		\end{figure}
	\end{pwrblock}
\end{frame}

% % % % % % % % % % % % % % % % % % % % % % % % % % % % % 
% Frame 11
% % % % % % % % % % % % % % % % % % % % % % % % % % % % %


\begin{frame}%[allowframebreaks=.7]
	\frametitle{Problem Incremental - czas dzia�ania}
	
	\begin{pwrblock}[Algorytmy iteracyjne a model \textsc{LP}]
		\begin{figure}[htbp]
			\centering
			\pgfimage[width=75mm,interpolate=false]{frame11/IMST1psfrag}
		\end{figure}
	\end{pwrblock}
\end{frame}

% % % % % % % % % % % % % % % % % % % % % % % % % % % % % 
% Frame 12
% % % % % % % % % % % % % % % % % % % % % % % % % % % % %


\begin{frame}%[allowframebreaks=.7]
	\frametitle{Problem Incremental - czas dzia�ania}
	
	\begin{pwrblock}[Algorytmy iteracyjne a model \textsc{LP}]
		\begin{figure}[htbp]
			\centering
			\pgfimage[width=75mm,interpolate=false]{frame12/IMST3psfrag}
		\end{figure}
	\end{pwrblock}
\end{frame}

% % % % % % % % % % % % % % % % % % % % % % % % % % % % % 
% Frame 13
% % % % % % % % % % % % % % % % % % % % % % % % % % % % %


\begin{frame}%[allowframebreaks=.7]
	\frametitle{Problem Recoverable - idea}
	
	\begin{pwrblock}[Min-Max]
		\begin{gather*}
			\min_{\mathclap{\textbf{x} \in X}} \max_{\mathclap{\textsc{s} \in S}} v \left( \textbf{x}, \textbf{s} \right) \\
			\textbf{s} \in S = \left\{ \textbf{s}^{1}, \textbf{s}^{2}, \dots, \textbf{s}^{n} \right\}
		\end{gather*}
	\end{pwrblock}
	
	\begin{pwrblock}[Recoverable Robust]
	\begin{gather*}
	\min_{\mathclap{\textbf{x} \in X}} \left( v \left( \textbf{x}, \textbf{s} \right) + \max_{\mathclap{\textbf{s}^{\prime} \in S}} \min_{\mathclap{\textbf{y} \in S_{\textbf{x}}}} v \left( \textbf{y}, \textbf{s}^{\prime} \right) \right)
	\end{gather*}
	\end{pwrblock}
\end{frame}

% % % % % % % % % % % % % % % % % % % % % % % % % % % % % 
% Frame 14
% % % % % % % % % % % % % % % % % % % % % % % % % % % % %


\begin{frame}%[allowframebreaks=.7]
	\frametitle{Problem Recoverable dla IMST - idea}

	\begin{pwrblock}[Problem adwersarza]
		\begin{gather*}
		\max_{\mathclap{\textbf{s}^{\prime} \in S}} \min_{\mathclap{\textbf{y} \in S_{\textbf{x}}}} v \left( \textbf{y}, \textbf{s}^{\prime} \right)
		\end{gather*}
	\end{pwrblock}
	
	\begin{pwrblock}[Recoverable Robust Incr. Minimum Spanning Tree]
		\begin{gather*}
		\min_{\mathclap{T \in \mathcal{T} \left( G \right)}} \; \left( \sum_{e \in T} c_{e}^{\textbf{s}} + \max_{\mathclap{\textbf{s}^{\prime} \in S}} \quad \quad \min_{\mathclap{T^{\prime} \in \mathcal{T} \left( G_{\textbf{s}^{\prime}}, T, k \right)}} \quad \quad \sum_{e^{\prime} \in T^{\prime}} c_{e^{\prime}}^{\textbf{s}^{\prime}} \right)
		\end{gather*}
	\end{pwrblock}
\end{frame}

% % % % % % % % % % % % % % % % % % % % % % % % % % % % % 
% Frame 15
% % % % % % % % % % % % % % % % % % % % % % % % % % % % %


\begin{frame}%[allowframebreaks=.7]
	\frametitle{Tabu Search}
	
	\begin{pwrblock}[Recoverable Robust Incr. Minimum Spanning Tree]
		Problem \textsc{NP}-trudny, nieaproksymowalny.
		
		Dow�d poprzez sprowadzenie problemu decyzyjnego \textsc{Minimum Degree Spanning Tree} (\textsc{NP}-trudny) do \textsc{RRIMST}.	
	\end{pwrblock}
	\vspace{10pt}
	\textit{Adam Kasperski, Adam Kurpisz, and Pawe� Zieli�ski. Recoverable Robust Combinatorial Optimization Problems, strony $147$-�$153$. Springer International Publishing, Cham, 2014.}
	
\end{frame}

% % % % % % % % % % % % % % % % % % % % % % % % % % % % % 
% Frame 16
% % % % % % % % % % % % % % % % % % % % % % % % % % % % %


\begin{frame}%[allowframebreaks=.7]
	\frametitle{Tabu Search - idea}
	
	\begin{pwrblock}[Kroki]
		\begin{itemize}
			\item We� losowy punkt z przestrzeni dopuszczalnych rozwi�za�,
			\pause
			\item wykonaj \textbf{ruch} w \textbf{otoczeniu} obecnego rozwi�zania na podstawie \textbf{funkcji oceny ruchu},
			\pause
			\item powtarzaj a� nie zajdzie pewien warunek,
			\pause
			\item wybierz inne losowe rozwi�zanie, dalekie od tego, wybranego na pocz�tku,
			\pause
			\item powtarzaj powy�sze kroki a� do momentu, gdy zajdzie pewien warunek ko�ca.
		\end{itemize}	
	\end{pwrblock}
\end{frame}
	
	
	
	

% % % % % % % % % % % % % % % % % % % % % % % % % % % % % 
% Frame 17
% % % % % % % % % % % % % % % % % % % % % % % % % % % % %


\begin{frame}%[allowframebreaks=.7]
	\frametitle{Tabu Search - wyniki}
	
	\begin{pwrblock}[Funkcja oceny ruchu - r�nica warto�ci rozwi�za�]
		\begin{figure}[htbp]
			\centering
			\pgfimage[width=75mm,interpolate=false]{frame17/RRIMST2psfrag}
		\end{figure}
	\end{pwrblock}
\end{frame}




% % % % % % % % % % % % % % % % % % % % % % % % % % % % % 
% Frame 18
% % % % % % % % % % % % % % % % % % % % % % % % % % % % %


\begin{frame}%[allowframebreaks=.7]
	\frametitle{Tabu Search - wyniki}
	
	\begin{pwrblock}[$k = 1$]
		\begin{figure}[htbp]
			\centering
			\pgfimage[width=75mm,interpolate=false]{frame18/RRIMST5psfrag}
		\end{figure}
	\end{pwrblock}
\end{frame}

% % % % % % % % % % % % % % % % % % % % % % % % % % % % % 
% Frame 19
% % % % % % % % % % % % % % % % % % % % % % % % % % % % %


\begin{frame}%[allowframebreaks=.7]
	\frametitle{Tabu Search - wyniki}
	
	\begin{pwrblock}[$k = 7$]
		\begin{figure}[htbp]
			\centering
			\pgfimage[width=75mm,interpolate=false]{frame19/RRIMST6psfrag}
		\end{figure}
	\end{pwrblock}
\end{frame}

% % % % % % % % % % % % % % % % % % % % % % % % % % % % % 
% Frame 20
% % % % % % % % % % % % % % % % % % % % % % % % % % % % %


\begin{frame}%[allowframebreaks=.7]
	\frametitle{Tabu Search - wyniki}
	
	\begin{pwrblock}[]
		\begin{figure}[htbp]
			\centering
			\pgfimage[width=75mm,interpolate=false]{frame20/RRIMST10psfrag}
		\end{figure}
	\end{pwrblock}
\end{frame}

\begin{frame}%[allowframebreaks=.7]
	\frametitle{Tabu Search - wyniki}
	
	\begin{table}
		\centering
		\begin{tabular}{ccc}
			\hline
			\# & $v_{\textsc{ts}} \left( T, S \right)$ & $\frac{v_{\textsc{ts}} \left( T, S \right) - v_{\textsc{ts}}^{LB^{\ast}}}{v_{\textsc{ts}}^{LB^{\ast}}}$ \\
			$1$	&	$5388$	&	$11,28\%$	\\
			$3$	&	$5151$	&	$6,38\%$	\\
			$8$	&	$4996$	&	$3,18\%$	\\
			$10$	&	$4988$	&	$3,02\%$	\\
			$13$	&	$4941$	&	$2,04\%$	\\
			$40$	&	$4927$	&	$1,76\%$	\\
			$116$	&	$4880$	&	$0,78\%$	\\
			$584$	&	$4851$	&	$0,19\%$ \\\hline                                                                                                    
		\end{tabular}
	\end{table}
\end{frame}

% % % % % % % % % % % % % % % % % % % % % % % % % % % % % 
% Frame 22
% % % % % % % % % % % % % % % % % % % % % % % % % % % % %


\begin{frame}%[allowframebreaks=.7]
	\frametitle{Tabu Search - parametry}
	
	\begin{pwrblock}[Funkcje oceny ruchu]
		\begin{gather*}
		Mval \left( T, T_{1} \right) = v_{\textsc{RRIMST}} \left( T, S \right) - v_{\textsc{RRIMST}} \left( T_{1}, S \right)\text{,}\\
		\pause
		Mval \left( T, T_{1} \right) = \alpha_{1} \cdot \left( v_{\textsc{RRIMST}} \left( T, S \right) - v_{\textsc{RRIMST}} \left( T_{1}, S \right) \right) +\\
		+ \alpha_{2} \cdot \frac{\text{R} \left[ i, j \right]}{it} + \alpha_{3} \cdot \frac{\text{R} \left[ k, l \right]}{it} + \alpha_{4} \cdot \text{MR} \left[ i, j \right] + \alpha_{5} \cdot \text{MR} \left[ k, l \right]
		\end{gather*}
	\end{pwrblock}
\end{frame}

\begin{frame}%[allowframebreaks=.7]
	\frametitle{Tabu Search - parametry}

	\begin{pwrblock}[Dolne ograniczenie]
		\begin{gather*}
		 \left| S \right| = 1 \rightarrow \min_{\mathclap{\textbf{x} \in X}} v \left( \textbf{x}, \textbf{s} \right) + \max_{\mathclap{\textbf{s}^{\prime} \in S}} \min_{\mathclap{\textbf{y} \in X^{k}_{\textbf{x}}}} v \left( \textbf{y}, \textbf{s}^{\prime} \right) \equiv\\
		 \equiv \min_{\mathclap{\textbf{x} \in X}} v \left( \textbf{x}, \textbf{s} \right) + \min_{\mathclap{\textbf{y} \in X^{k}_{\textbf{x}}}} v \left( \textbf{y}, \textbf{s}^{\prime} \right)\text{,}\\
		 \textsc{LB} \leqslant \textsc{LB}^{\ast} = \max_{\mathclap{\textsc{LB} \in \mathcal{LB}}} \textsc{LB} \leqslant v_{\textsc{RRIMST}}^{\ast}\text{.}
		\end{gather*}
	\end{pwrblock}
\end{frame}

\end{document}