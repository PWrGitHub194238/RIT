% -*- TeX:PL -*-
% $Id: $
\documentclass[12pt]{beamer}
\usepackage[T1]{polski}
\usepackage[cp1250]{inputenc}
\usepackage{lipsum}
\usepackage{multimedia}

\usepackage{mathtools}
\usepackage[ruled,vlined]{algorithm2e}
%\usepackage[a4paper]{geometry}
%\usepackage{graphicx}
%\usepackage[]{hyperref}
%

%
\author{Tomasz Strza�ka}
\title{Wybrane problemy \\ odpornej optymalizacji dyskretnej}
 \subtitle{z mo�liwo�ci� modyfikacji}
 \date{\today}
 \institute{Wydzia� Podstawowych Problem�w Techniki}
 \subject{Logotyp Politechniki Wroc�awskiej}
 \keywords{logotyp, Politechnika Wroc�awska, LaTeX}

\setbeamersize{text margin left=0mm,text margin right=2.5mm}
%\setbeamersize{sidebar width left=0mm,sidebar width right=0cm}
\usetheme[lang=pl,pasek=pasek1]{pwr}

\setbeamercovered{transparent=25}

\newtheorem{pwrblock}{Definicja}
\newenvironment<>{pwrblock}[1][]{%
	\setbeamertemplate{blocks}[rounded][shadow=true]
	\setbeamercolor{block title}{fg=white,bg=red!75!black}%
	\begin{block}#2{#1}}{\end{block}}


\transdissolve[duration=0.2]
\logo{\includegraphics[height=7mm]{poziom-pl-pantone}}
\logo{\includegraphics[height=7.05mm]{logopwrpl}}

\begin{document}
\begin{frame}[plain,t]
 \maketitle
\end{frame}

% % % % % % % % % % % % % % % % % % % % % % % % % % % % % 
% Frame 1
% % % % % % % % % % % % % % % % % % % % % % % % % % % % %

\begin{frame}%[allowframebreaks=.7]
	\frametitle{Minimalne drzewo rozpinaj�ce}
	
	\begin{minipage}{0.5\textwidth}
		\pgfimage[width=40mm,interpolate=false]{frame1/mst}
	\end{minipage}% This must go next to `\end{minipage}
	\begin{minipage}{0.5\textwidth}
		\begin{pwrblock}[W�a�ciwo�ci]
			\begin{gather*}
				T^{\ast} = \min_{\mathclap{T \in \mathcal{T}}} c \left( T \right) \\
				\left| T \right| = \left| V \right| - 1 \\
				\exists v \overset{\ast}{\leadsto} v^{\prime} \; : \; v \neq v^{\prime}
			\end{gather*}
		\end{pwrblock}
	\end{minipage}

\end{frame}

% % % % % % % % % % % % % % % % % % % % % % % % % % % % % 
% Frame 2
% % % % % % % % % % % % % % % % % % % % % % % % % % % % %


\begin{frame}%[allowframebreaks=.7]
	\frametitle{Przyk�ad}
	
	\begin{figure}[htbp]
		\centering
		\pgfimage[width=50mm,interpolate=false]{frame2/mstExample}
	\end{figure}
	
\end{frame}

% % % % % % % % % % % % % % % % % % % % % % % % % % % % % 
% Frame 3
% % % % % % % % % % % % % % % % % % % % % % % % % % % % %


\begin{frame}%[allowframebreaks=.7]
	\frametitle{Przyk�ad}
	
	\begin{figure}[htbp]
		\centering
		\pgfimage[width=115mm,interpolate=false]{frame3/mstExample}
	\end{figure}
	
	\begin{pwrblock}[W�a�ciwo�ci]
		Graf pe�ny z $43$ wierzcho�kami.
	\end{pwrblock}
	
\end{frame}


% % % % % % % % % % % % % % % % % % % % % % % % % % % % % 
% Frame 4
% % % % % % % % % % % % % % % % % % % % % % % % % % % % %


\begin{frame}%[allowframebreaks=.7]
	\frametitle{Problem Incremental}
	
	\begin{pwrblock}[Opis]
		\begin{gather*}
			\min_{\mathclap{\textbf{y} \in S^{k}_{\textbf{x}}}} v \left( \textbf{y}, \textbf{s} \right) \\
			\textbf{x} \in X \\
			\textbf{x}^{\ast} = \min_{\mathclap{\textbf{x} \in X}}arg \; v \left( \textbf{x}, \textbf{s}^{0} \right) \\
			\textbf{s} \in S = \left\{ \textbf{s}^{1}, \textbf{s}^{2}, \dots, \textbf{s}^{n} \right\} \\
			v \left( \textbf{x}, \textbf{s}^{0} \right) = \sum_{e_{i} \in E} c_{i}^{\textbf{s}} \cdot  x_{i}
		\end{gather*}
	\end{pwrblock}
\end{frame}


% % % % % % % % % % % % % % % % % % % % % % % % % % % % % 
% Frame 5
% % % % % % % % % % % % % % % % % % % % % % % % % % % % %


\begin{frame}%[allowframebreaks=.7]
	\frametitle{Problem minimaxowy}
	
	\begin{pwrblock}[Opis]
		\begin{gather*}
			\min_{\mathclap{\textbf{x} \in X}} \max_{\mathclap{\textsc{s} \in S}} v \left( \textbf{x}, \textbf{s} \right) \\
			\textbf{s} \in S = \left\{ \textbf{s}^{1}, \textbf{s}^{2}, \dots, \textbf{s}^{n} \right\}
		\end{gather*}
	\end{pwrblock}
\end{frame}


% % % % % % % % % % % % % % % % % % % % % % % % % % % % % 
% Frame 6
% % % % % % % % % % % % % % % % % % % % % % % % % % % % %


\begin{frame}%[allowframebreaks=.7]
	\frametitle{Najgorszy scenariusz}
	
	\begin{figure}[htbp]
		\centering
		\pgfimage[width=68mm,interpolate=false]{frame6/BlackHole}
	\end{figure}
	
	\begin{pwrblock}[Opis]
		\centering
		$\textbf{s}^{?}$ --- pole grawitacyjne czarnej dziury zniekszta�ca orbity planet uk�adu s�onecznego powoduj�c kolizje mi�dzy nimi.
		($Pr \left[ \textbf{s}^{?} \right] \approx \frac{1}{10^{18}}$)
	\end{pwrblock}
\end{frame}


% % % % % % % % % % % % % % % % % % % % % % % % % % % % % 
% Frame 7
% % % % % % % % % % % % % % % % % % % % % % % % % % % % %


\begin{frame}%[allowframebreaks=.7]
	\frametitle{Regret min-max}
	
	\begin{pwrblock}[Opis]
		\begin{gather*}
		\min_{\mathclap{\textbf{x} \in X}} \max_{\mathclap{\textsc{s} \in S}} \left( v \left( \textbf{x}, \textbf{s} \right) - v^{\ast}_{\textbf{s}} \right) \\
		v^{\ast}_{\textbf{s}} = v \left( \textbf{x}^{\ast}_{\textbf{s}}, \textbf{s} \right) \\
		\textbf{x}^{\ast}_{\textbf{s}} = \min_{\mathclap{\textbf{x} \in X}}arg \; v \left( \textbf{x}, \textbf{s} \right)
		\end{gather*}
	\end{pwrblock}
\end{frame}

% % % % % % % % % % % % % % % % % % % % % % % % % % % % % 
% Frame 8
% % % % % % % % % % % % % % % % % % % % % % % % % % % % %


\begin{frame}%[allowframebreaks=.7]
	\frametitle{Najgorszy scenariusz}
	
	\begin{figure}[htbp]
		\centering
		\pgfimage[width=68mm,interpolate=false]{frame6/BlackHole}
	\end{figure}
	
	\begin{pwrblock}[Opis]
		\begin{equation*}
			\forall \textbf{x} \in X \; v \left( \textbf{x}, \textbf{s} \right) - v^{\ast}_{\textbf{s}}\text{ --- ,,ma�e''}
		\end{equation*}
	\end{pwrblock}
\end{frame}
	


% % % % % % % % % % % % % % % % % % % % % % % % % % % % % 
% Frame 9
% % % % % % % % % % % % % % % % % % % % % % % % % % % % %


\begin{frame}%[allowframebreaks=.7]
	\frametitle{Problem Incremental}
	
	\begin{pwrblock}[Opis]
		\begin{gather*}
			\min_{\mathclap{\textbf{y} \in S^{k}_{\textbf{x}}}} v \left( \textbf{y}, \textbf{s} \right)
		\end{gather*}
	\end{pwrblock}
\end{frame}


% % % % % % % % % % % % % % % % % % % % % % % % % % % % % 
% Frame 10
% % % % % % % % % % % % % % % % % % % % % % % % % % % % %


\begin{frame}%[allowframebreaks=.7]
	\frametitle{Odporna optymalizacja}
	
	\begin{pwrblock}[Opis]
		\begin{gather*}
			\min_{\mathclap{\textbf{x} \in X}} \left( v \left( \textbf{x}, \textbf{s} \right) + \max_{\mathclap{\textbf{s}^{\prime} \in S}} \min_{\mathclap{\textbf{y} \in S_{\textbf{x}}}} v \left( \textbf{y}, \textbf{s}^{\prime} \right) \right)
		\end{gather*}
	\end{pwrblock}
\end{frame}


% % % % % % % % % % % % % % % % % % % % % % % % % % % % % 
% Frame 11
% % % % % % % % % % % % % % % % % % % % % % % % % % % % %


\begin{frame}%[allowframebreaks=.7]
	\frametitle{Odporna optymalizacja}
	
	\begin{pwrblock}[Problem MST]
		\begin{gather*}
			\min_{\mathclap{\textbf{y}}} v \left( \textbf{y}, \textbf{s} \right)
		\end{gather*}
	\end{pwrblock}
	
	\begin{pwrblock}[Problem Incremental MST]
		\begin{gather*}
		\min_{\mathclap{\textbf{y} \in S^{k}_{\textbf{x}}}} v \left( \textbf{y}, \textbf{s} \right)
		\end{gather*}
	\end{pwrblock}

\end{frame}


% % % % % % % % % % % % % % % % % % % % % % % % % % % % % 
% Frame 12
% % % % % % % % % % % % % % % % % % % % % % % % % % % % %


\begin{frame}%[allowframebreaks=.7]
	\frametitle{Odporna optymalizacja}
	
	\begin{pwrblock}[Problem Adwersarza MST]
		\begin{gather*}
		\max_{\mathclap{\textbf{s}^{\prime} \in S}} \min_{\mathclap{\textbf{y} \in S_{\textbf{x}}}} v \left( \textbf{y}, \textbf{s}^{\prime} \right)
		\end{gather*}
	\end{pwrblock}
	
	\begin{pwrblock}[Odporna optymalizacja MST]
		\begin{gather*}
		\min_{\mathclap{\textbf{x} \in X}} \left( v \left( \textbf{x}, \textbf{s} \right) + \max_{\mathclap{\textbf{s}^{\prime} \in S}} \min_{\mathclap{\textbf{y} \in S_{\textbf{x}}}} v \left( \textbf{y}, \textbf{s}^{\prime} \right) \right)
		\end{gather*}
	\end{pwrblock}
	
\end{frame}



% % % % % % % % % % % % % % % % % % % % % % % % % % % % % 
% Frame 13
% % % % % % % % % % % % % % % % % % % % % % % % % % % % %


\begin{frame}%[allowframebreaks=.7]
	\frametitle{Incremental LP}
	\small
	\begin{subequations}
		\begin{alignat*}{4}
		& \text{min} & & \sum_{ e \in E} c_{e} \cdot y_{e}, \\
		& \text{s.t.} & \quad & \sum_{\mathclap{ \left( j, s \right ) \in E }} f^{k}_{js} - \sum_{\mathclap{ \left( s, j \right ) \in E }} f^{k}_{sj} = -1, & \forall k \in V \setminus \left\{ v_{s} \right\},\\
		& & & \sum_{\mathclap{ \left( j, i \right ) \in E }} f^{k}_{ji} - \sum_{\mathclap{ \left( i, j \right ) \in E }} f^{k}_{ij} = 0, & \forall i, k \in V \setminus \left\{ v_{s} \right\} \; \wedge \; i \neq k,\\
		& & & \sum_{\mathclap{ \left( j, k \right ) \in E }} f^{k}_{jk} - \sum_{\mathclap{ \left( k, j \right ) \in E }} f^{k}_{kj} = 1, & \forall k \in V \setminus \left\{ v_{s} \right\},\\
		& & & f^{k}_{ij} \leqslant y_{ij}, & \forall \left( i, j \right) \in E \; \wedge \; \forall k \in V \setminus \left\{ v_{s} \right\},&&&\\
		& & & \sum_{\mathclap{ \left( i, j \right) \in E}} y_{ij} = n - 1, & \quad & &\\
		& & & \phantom{\sum} f_{ij} \geqslant 0, & \forall \left( i, j \right) \in E,&&&\\
		& & & \phantom{\sum} y_{ij} \geqslant 0, & \forall \left( i, j \right) \in E.&&&
		\end{alignat*}
	\end{subequations}
\end{frame}


% % % % % % % % % % % % % % % % % % % % % % % % % % % % % 
% Frame 14
% % % % % % % % % % % % % % % % % % % % % % % % % % % % %


\begin{frame}%[allowframebreaks=.7]
	\frametitle{Incremental LP}
	\small
	\begin{subequations}
		\begin{alignat*}{6}
		& \text{s.t.} & \cdots &&\\
		& & & f^{k}_{ij} \leqslant y_{ij}, & \forall \left( i, j \right) \in E \; \wedge \; \forall k \in V \setminus \left\{ v_{s} \right\},&&&\\
		& & & \sum_{\mathclap{ \left( i, j \right) \in E}} y_{ij} = n - 1, & \quad & &\\
		& & & \sum_{\left( i, j \right) \in E} \left| x_{ij} - y_{ij} \right| \leqslant k &&&&\\
		& & & \sum_{\mathclap{ \left( i, j \right) \in E}} y_{ij} = n - 1, & \quad & &\\
		& & & x_{ij} = 1, & \forall \left( i, j \right) \in T^{0} & &\\
		& & & x_{ij} = 0, & \forall \left( i, j \right) \notin T^{0} & &\\
		\end{alignat*}
	\end{subequations}
\end{frame}


% % % % % % % % % % % % % % % % % % % % % % % % % % % % % 
% Frame 15
% % % % % % % % % % % % % % % % % % % % % % % % % % % % %


\begin{frame}%[allowframebreaks=.7]
	\frametitle{Incremental LP}
	\small
	\begin{subequations}
		\begin{alignat*}{3}
		& \text{s.t.} & \cdots &&\\
		& \phantom{\sum} f_{ij} \geqslant 0, & \forall \left( i, j \right) \in E,&&\\
		& \phantom{\sum} y_{ij} \geqslant 0, & \forall \left( i, j \right) \in E.&&
		\end{alignat*}
	\end{subequations}
\end{frame}



% % % % % % % % % % % % % % % % % % % % % % % % % % % % % 
% Frame 16
% % % % % % % % % % % % % % % % % % % % % % % % % % % % %


\begin{frame}%[allowframebreaks=.7]
	\frametitle{Incremental LP}
	
	\begin{subequations}
		\begin{alignat*}{4}
		& \text{min} & & \sum_{e_{i} \in E^{\ast}} c_{i} \cdot x_{i} \\
		& \text{s.t.} & \quad & \sum_{\mathclap{e_{i} \in E^{\ast}}} x_{i} = \left| V \right| - 1,\\
		& & & \sum_{\mathclap{e_{i} \in E^{\ast} \left( S \right) }} x_{i} = \left| S \right| - 1, & \quad & S & \subseteq V,\\
		& & & \phantom{\sum} x_{i} \geqslant 0, &\quad & e_{i} &\in E^{\ast}, \\
		& & & \sum_{e_{i} \in T^{\ast} \setminus T^{0}} x_{i} \leqslant k.
		\end{alignat*}
	\end{subequations}
	
\end{frame}


% % % % % % % % % % % % % % % % % % % % % % % % % % % % % 
% Frame 17
% % % % % % % % % % % % % % % % % % % % % % % % % % % % %


\begin{frame}%[allowframebreaks=.7]
	\frametitle{Incremental LP}
	
	\begin{subequations}
		\begin{alignat*}{4}
		& \text{min} & & \sum_{e_{i} \in E^{\ast}} c_{i} \cdot x_{i} \\
		& \text{s.t.} & \quad & \sum_{\mathclap{e_{i} \in E^{\ast}}} x_{i} = \left| V \right| - 1,\\
		& & & \sum_{\mathclap{e_{i} \in E^{\ast} \left( S \right) }} x_{i} = \left| S \right| - 1, & \quad & S & \subseteq V,\\
		& & & \phantom{\sum} x_{i} \geqslant 0, &\quad & e_{i} &\in E^{\ast}, \\
		& & & \sum_{\mathclap{e_{i} \in T^{0}}} x_{i} \geqslant n - 1 - k.
		\end{alignat*}
	\end{subequations}
	
\end{frame}


% % % % % % % % % % % % % % % % % % % % % % % % % % % % % 
% Frame 18
% % % % % % % % % % % % % % % % % % % % % % % % % % % % %


\begin{frame}%[allowframebreaks=.7]
	\frametitle{Incremental LP - relaksacja}
	
	\begin{subequations}
		\begin{alignat*}{4}
		& \text{min} & & \sum_{e_{i} \in E^{\ast}} c_{i} \cdot x_{i} + \lambda \cdot \left( \left( n - 1 - k \right) - \sum_{\mathclap{e_{i} \in T^{0}}} x_{i} \right) \\
		& \text{s.t.} & \quad & \sum_{\mathclap{e_{i} \in E^{\ast}}} x_{i} = \left| V \right| - 1,\\
		& & & \sum_{\mathclap{e_{i} \in E^{\ast} \left( S \right) }} x_{i} = \left| S \right| - 1, & \quad & S & \subseteq V,\\
		& & & \phantom{\sum} x_{i} \geqslant 0, &\quad & e_{i} &\in E^{\ast}.
		\end{alignat*}
	\end{subequations}
	
\end{frame}


% % % % % % % % % % % % % % % % % % % % % % % % % % % % % 
% Frame 19
% % % % % % % % % % % % % % % % % % % % % % % % % % % % %


\begin{frame}%[allowframebreaks=.7]
	\frametitle{Incremental LP - funkcja celu}
	
	\begin{eqnarray*}
	L \left( \lambda, x \right) = \sum_{e_{i} \in E^{\ast}} c_{i} \cdot x_{i} + \lambda \cdot \left( \left( n - 1 - k \right) - \sum_{\mathclap{e_{i} \in T^{0}}} x_{i} \right) = \\
	\sum_{e_{i} \in T^{\ast} \cup T^{0}} c_{i} \cdot x_{i} - \lambda \sum_{\mathclap{e_{i} \in T^{0}}} x_{i} + \lambda \cdot \left( n - 1 - k \right) = \cdots \\ \cdots = \min \left\{ \sum_{e_{i} \in T^{\ast} \setminus T^{0}} c_{i} \cdot x_{i} + \sum_{\mathclap{e_{i} \in T^{0}}} \left( c_{i} - \lambda \right) \cdot x_{i} \right\}
	\end{eqnarray*}
	
\end{frame}


% % % % % % % % % % % % % % % % % % % % % % % % % % % % % 
% Frame 20
% % % % % % % % % % % % % % % % % % % % % % % % % % % % %


\begin{frame}%[allowframebreaks=.7]
	\frametitle{Tabu Search}
	
\end{frame}

\end{document}