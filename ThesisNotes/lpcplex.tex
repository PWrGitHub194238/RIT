\chapter{The robust spanning tree problem with interval data}
\thispagestyle{chapterBeginStyle}


Modele LP dla MST:

Pierwszy model jest czysto teoretyczny, opiera się na definicji. W każdym podzbiorze krawędzi wierzchołków $S$ o mocy $\left| S \right|$ ma występować dokładnie $\left| S \right| - 1$ krawędzi. Czyli:

\begin{eqnarray}
	E \left( S \right) = \left\{ \left\{ i, j \right\} : v_{i} \overset{1}\leadsto v_{j} \in E \; \wedge \; v_{i} \in S \; \wedge \; v_{j} \in S \right\} \\
	S \subseteq V \\
	E \left( V \right) = E \\
	\sum_{e \in E} x_{e} = \left| V \right| - 1\\
	\sum_{e \in E \left( S \right)} x_{e} = \left| S \right| - 1 \; \forall S \subseteq V \\
	\forall e \in E \; x_{e} \in \left\{ 0, 1 \right\}
\end{eqnarray}

Poniższe modele można śmiało kopiować do pracy:

\begin{subequations}
	\begin{alignat}{4}
		& \text{min} & & \sum_{\mathclap{e \in E}} c_{e} \cdot x_{e}, \\
		& \text{s.t.} & \quad & \sum_{\mathclap{e_ \in E}} x_{e} = \left| V \right| - 1,\\
		& & & \sum_{\mathclap{e \in E \left( S \right) }} x_{e} = \left| S \right| - 1, & \quad & S & \subseteq V,\\
		& & & \phantom{\sum} x_{e} \geqslant 0, &\quad & e &\in E.
	\end{alignat}
\end{subequations}

lub w formie:

\begin{equation}
\begin{aligned}
	& \text{min} & & \sum_{\mathclap{e \in E}} c_{e} \cdot x_{e}, \\
	& \text{s.t.} & \quad & \sum_{\mathclap{e_ \in E}} x_{e} = \left| V \right| - 1,\\
	& & & \sum_{\mathclap{e \in E \left( S \right) }} x_{e} = \left| S \right| - 1, & \quad & S & \subseteq V,\\
	& & & \phantom{\sum} x_{e} \geqslant 0, &\quad & e &\in E.
\end{aligned}
\end{equation}

Drugi model opiera się na przepływie, w punkcie początkowym $v_{1}$ mamy $n - 1$ jednostek, do każdego wierzchołka innego niż $v_{1}$. Dla każdego wierzchołka poza początkowym zatem suma jednostek wychodzących ma być większa o dokładnie $1$ niż jednostek przychodzących ($1$ jednostkę wierzchołek sobie zatrzymuje, resztę przekazuje dalej). Jeśli dana krawędź $e_{ij}$ należy do rozwiązania to jej flow może wynosić max $n - 1$. Graf jest nieskierowany, więc to samo dla $f_{ij}$ i dla $f_{ji}$. Liczba krawędzi ma być równa $n-1$.

\begin{subequations}
	\begin{alignat}{4}
	& \text{min} & & \sum_{\mathclap{e \in E}} c_{e} \cdot x_{e}, \\
	& \text{s.t.} & \quad & \sum_{\mathclap{ \left( i, j \right ) \in E }} f_{ij} - \sum_{\mathclap{ \left( j, i \right ) \in E }} f_{ji} &= \left\{
		\begin{matrix}
			n - 1 & \text{jeżeli}~ i = 1 \text{,}\\ 
			-1 & \text{w przeciwnym przypadku,}
		\end{matrix}
	\right.\\
	& & & f_{ij} \leqslant \left( n - 1 \right) \cdot x_{ij}, & \forall  \left\{ i, j \right\} \in E,&&&\\
	& & & f_{ji} \leqslant \left( n - 1 \right) \cdot x_{ij}, & \forall  \left\{ i, j \right\} \in E,&&&\\
	& & & \sum_{\mathclap{e \in E}} x_{e} = n - 1, & & &\\
	& & & \phantom{\sum} f_{e} \geqslant 0, & \forall e \in E,&&&\\
	& & & \phantom{\sum} x_{e} \in \left\{ 0, 1 \right\}, & \forall e\in E.&&&
	\end{alignat}
\end{subequations}


Trzeci schemat rozwarstwia graf. Teraz każda jednostka przepływu płynie na swojej warstwie. Dodanie jeszcze jednego wymiaru skutkuje w unimodularności macierzy.

Teraz każdy wierzchołek poza początkowym ma jedną konkretną jednostkę $k$, którą musi dostać. Zatem dla każdej takiej jednostki suma wychodzących jednostek ze źródła musi być większa o 1 niż jednostek wchodzących do źródła. Dla każdego wierzchołka pośredniego $i$, jeżeli to nie jego jednostka, różnica sumy wejścia do $i$ i wyjścia z $i$ ma być równa $0$ --- węzeł przepuszcza nie swoją jednostkę.

Analogicznie dla swojej jednostki. Dla każdej jednostki $k$, suma wchodzących jednostek do wierzchołka, dla którego ta jednostka jest przeznaczona ma być o $1$ większa niż suma jednostek z niego wychodzących --- węzeł zatrzymuje swoją jednostkę.

Każdy flow $f^{k}_{ij}$ jest teraz zerojedynkowy.

Mamy podziała na $y_{ij}$ i $y_{ji}$, bo poprzednio mieliśmy 2 warunki na flowy, tutaj tylko 1.

\begin{subequations}
	\begin{alignat}{4}
	& \text{min} & & \sum_{ \left( i, j \right) \in E} c_{ij} \cdot \left( y_{ij} + y_{ji} \right), \\
	& \text{s.t.} & \quad & \sum_{\mathclap{ \left( j, s \right ) \in E }} f^{k}_{js} - \sum_{\mathclap{ \left( s, j \right ) \in E }} f^{k}_{sj} = -1, & \forall k \in V \setminus \left\{ v_{s} \right\},\\
	& & & \sum_{\mathclap{ \left( j, i \right ) \in E }} f^{k}_{ji} - \sum_{\mathclap{ \left( i, j \right ) \in E }} f^{k}_{ij} = 0, & \forall i, k \in V \setminus \left\{ v_{s} \right\} \; \wedge \; i \neq k,\\
	& & & \sum_{\mathclap{ \left( j, k \right ) \in E }} f^{k}_{jk} - \sum_{\mathclap{ \left( k, j \right ) \in E }} f^{k}_{kj} = 1, & \forall k \in V \setminus \left\{ v_{s} \right\},\\
	& & & f^{k}_{ij} \leqslant y_{ij}, & \forall \left( i, j \right) \in E \; \wedge \; \forall k \in V \setminus \left\{ v_{s} \right\},&&&\\
	& & & \sum_{\mathclap{ \left( i, j \right) \in E}} y_{ij} = n - 1, & \quad & &\\
	& & & \phantom{\sum} f_{ij} \geqslant 0, & \forall \left( i, j \right) \in E,&&&\\
	& & & \phantom{\sum} y_{ij} \geqslant 0, & \forall \left( i, j \right) \in E.&&&
	\end{alignat}
\end{subequations}

Jak to przekuć na incremental:

rozdzielić zmienne decyzyjne wierzchołków na te, które są w starym rozwiązaniu i dodać warunek $\sum_{e \in E} \left| x_{e} - y_{e} \right| \leqslant K$. Wartość bezwzględną można zrobić poprzez rozbicie zmiennych na dodatnie i ujemne np. tak:

\begin{subequations}
	\begin{alignat}{4}
	& \text{min} & & \sum_{ e \in E} c_{e} \cdot y_{e}, \\
	& \text{s.t.} & \quad & \sum_{\mathclap{ \left( j, s \right ) \in E }} f^{k}_{js} - \sum_{\mathclap{ \left( s, j \right ) \in E }} f^{k}_{sj} = -1, & \forall k \in V \setminus \left\{ v_{s} \right\},\\
	& & & \sum_{\mathclap{ \left( j, i \right ) \in E }} f^{k}_{ji} - \sum_{\mathclap{ \left( i, j \right ) \in E }} f^{k}_{ij} = 0, & \forall i, k \in V \setminus \left\{ v_{s} \right\} \; \wedge \; i \neq k,\\
	& & & \sum_{\mathclap{ \left( j, k \right ) \in E }} f^{k}_{jk} - \sum_{\mathclap{ \left( k, j \right ) \in E }} f^{k}_{kj} = 1, & \forall k \in V \setminus \left\{ v_{s} \right\},\\
	& & & f^{k}_{ij} \leqslant y_{ij}, & \forall \left( i, j \right) \in E \; \wedge \; \forall k \in V \setminus \left\{ v_{s} \right\},&&&\\
	& & & \sum_{\mathclap{ \left( i, j \right) \in E}} y_{ij} = n - 1, & \quad & &\\
	& & & \phantom{\sum} f_{ij} \geqslant 0, & \forall \left( i, j \right) \in E,&&&\\
	& & & \phantom{\sum} y_{ij} \geqslant 0, & \forall \left( i, j \right) \in E.&&&
	\end{alignat}
\end{subequations}
gdzie $x_{e}$ w tym przypadku to albo rozwiązanie odpowiadające $x_{e}$ z jednemu z pierwszych formuł albo $\left( y_{ij} + y_{ji} \right)$ dla trzeciego, dla $e = \left( i, j \right)$.


w każdym razie optymalne zawsze ma albo z+i = 0 albo z-i=0 (bo albo dodajemy albo odejmujemy krawedz, nie robimy tego jednoczesnie).