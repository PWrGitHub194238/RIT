\chapter{Orlin Network Flows}
\thispagestyle{chapterBeginStyle}

Jeśli w funkcji celu potrzebujemy iloczyn kosztów krawędzi to można te koszty do modelu zlogarytmować w myśl równości:

\begin{equation}
\log \left( x \cdot y \right) = \log \left( x \right) + \log \left( y \right)
\end{equation}

Jeśli mamy problem minimaxowy (koszt ścieżki z $i$ do $j$ jest równy maksymalnemu kosztowi krawędzi na tej ścieżce), gdzie chcemy zminimalizować koszt danej ścieżki to budując drzewo MST możemy natychmiastowo rozwiązać ten problem. Niech $P$ będzie ścieżką w MST od $p$ do $q$ i niech jej koszt wynosi $c_{ij}$ (krawędź $\left( i, j \right))$ jest na ścieżce i ma największy koszt spośród ścieżek należących do $P$). Usuwając z MST krawędź $\left( i, j \right)$ tworzą się dwa osobne zbiory $S$ i $\overline{S}$ (robimy cięcie na drzewie --- teraz początek ścieżki $p \in S$ a $q \in \overline{S}$). Dodatkowo dla każdego łuku $\left( k, l \right)$, gdzie $k \in S$ i $l \in \overline{S}$ zachodzi $c_{ij} \leqslant c_{kl}$ (bo jeśli nie to w MST poddrzewa $S$ i $\overline{S}$ byłyby połączone inną krawędzią, której koszt jest mniejszy od $\left( i, j \right)$ - więc pierwotne MST nie byłoby minimalne). Teraz wystarczy zauważyć, że krawędź $\left( i, j \right)$ to właśnie szukany minimalny maksymalny koszt danej ścieżki --- koszt krawędzi $\left( i, j \right)$ w ścieżce $P$ jest największy, lecz jest on najmniejszy spośród wszystkich krawędzi, które łączyć by mogły poddrzewa $s$ oraz $\overline{S}$ (powodując, że da się dojść z $p$ do $q$). Stąd wniosek, że MST zawiera rozwiązanie problemu minimaksowego.

Warunki optymalności:

\begin{itemize}
	\item $T^{\ast}$ jest MST wtedy i tylko wtedy, gdy dla każdej krawędzi $\left( i, j \right) \in T^{\ast}$, dla każdego $k \in S$, każdego $l \in \overline{S}$ takich, że $\left( k, l \right)$ zachodzi $c_{ij} \leqslant c_{kl}$, gdzie $S$ i $\overline{S}$ są podrzewami stworzonymi przez usunięcie z drzewa $T^{\ast}$ krawędzi $\left( i, j \right)$ (poprzez wykonanie cięcia na drzewie $T^{\ast}$).
	
	W jedną stronę to wiadomo, gdyby tak nie było to istniałaby taka krawędź $\left( k, l \right)$, która ma mniejszy koszt niż $\left( i, j \right)$, więc po wyrzuceniu krawędzi $\left( i, j \right)$, stworzeniu dwóch poddrzew $S$ i $\overline{S}$ oraz po ponownym połączeniu ich w jedno drzewo za pomocą krawędzi $\left( k, l \right)$ otrzymalibyśmy drzewo o mniejszym koszcie niż pierwotne $T^{\ast}$, co przeczy optymalności $T^{\ast}$.
	
	Żeby pokazać w drugą stronę (że jeżeli drzewo $T_{0}$ spełnia warunek optymalnego cięcia to jest MST) załóżmy, że spełnia warunek optymalności i nie jest MST. Niech $T^{\ast}$ będzie MST i $T_{0} \neq T^{\ast}$ ($T_{0}$ różni się od $T_{\ast}$ co najmniej jedną krawędzią). Z $T_{0}$ usuńmy krawędź $\left( i, j \right)$, której nie ma w $T^{\ast}$, tworząc jednocześnie podział $T_{0}$ na dwa poddrzewa $S$ i $\overline{S}$. Dodając tą krawędź do drzewa $T_{\ast}$ stworzymy w nim cykl (własność drzewa, że dodanie jednej krawędzi stworzy cykl), do którego należeć będzie krawędź $\left( i, j \right)$, gdzie $i \in S$, $j \in \overline{S}$ i w szczególności krawędź $\left( k, l \right)$, gdzie $k \in \overline{S}$ i $l \in S$ (kolejność nie ma znaczenia, jako że są to grafy nieskierowane, lecz podkreśla to fakt konieczności istnienia krawędzi, która pozwala z powrotem przejść do zbioru $S$ i zamknąć cykl). $T^{\ast}$ jest MST, więc $c_{kl} \leqslant c_{ij}$ (zamienienie krawędzi $\left( k, l \right) \in T^{\ast}$ na inną krawędź, także łączącą ze sobą poddrzewa $S$ i $\overline{S}$ nie polepszy nam kosztu). Z założenia natomiast $T_{0}$ spełnia warunek optymalnego cięcia, więc $c_{ij} \leqslant c_{kl}$. Z dwóch powyższych nierówności dostajemy $c_{ij} = c_{kl}$. Stąd mamy wniosek, że zamiana krawędzi $\left( i, j \right) \in T_{0}$ ($\left( i, j \right) \notin T^{\ast}$) na krawędź $\left( k, l \right) \in T^{\ast}$ nie zmieni kosztu całego drzewa. Nazwijmy tak powstałe drzewo $T_{1}$. W oczywisty sposób drzewo $T_{1}$ ma więcej krawędzi wspólnych z drzewem $T^{\ast}$ niż drzewo $T_{0}$ ($\left| T_{0} \setminus T^{\ast} \right| - 1 = \left| T_{1} \setminus T^{\ast} \right|$) przy jednoczesnym zachowaniu $C_{T_{0}} = C_{T_{1}}$. Powtarzając ten proces dojdziemy do tego, że dla pewnego $k > 0$ zachodzić będzie $\left| T_{k} \setminus T^{\ast} \right| = 0$ przy zachowaniu $C_{T_{0}} = C_{T_{1}} = \dots C_{T_{k}} = C_{T^{\ast}}$, co pokazuje, drzewo $T_{0}$ jest także MST (choć niekoniecznie tym samym).
	
	\item $T^{\ast}$ jest MST wtedy i tylko wtedy, gdy dla każdej krawędzi $\left( k, l\right)$ nie będącej w tym drzewie, dla każdej krawędzi $\left( i, j \right)$ znajdującej się na ścieżce z $k$ do $l$ zachodzi $c_{ij} \leqslant c_{kl}$.
	
	W jedną stronę jest znowu prosto. Zakładamy, że $T^{\ast}$ jest MST i istnieje krawędź nienależąca do $T^{\ast}$ $\left( k, l \right)$ taka, że $c_{ij} > c_{kl}$. Gdy dołączymy krawędź $\left( k, l \right)$ do drzewa, powstanie wtedy cykl (no bo wszystkie krawędzie, które są na ścieżce z $k$ do $l$ wraz z tą krawędzią tworzą po prostu cykl), w którym nowo dodana krawędź będzie miała mniejszy koszt od przynajmniej jednej z pozostałych krawędzi z tego cyklu, więc pozbywając się tej krawędzi otrzymujemy drzewo o mniejszym koszcie, mamy sprzeczność.
	
	W drugą stronę też jest prosto. Bierzemy drzewo $T^{\ast}$, które jest MST, tworzymy dwa poddrzewa $S$ i $\overline{S}$ poprzez usunięcie z drzewa $T^{\ast}$ krawędzi $\left( i, j \right)$. Weźmy teraz dowolną ścieżkę z $k$ do $l$ taką, że $k \in S$ oraz $l \in \overline{S}$. Jasnym jest, że skoro usuwając krawędź z $\left( i, j \right)$ drzewa $T^{\ast}$ rozdzieliliśmy drzewo na dwa zbiory, gdzie początek ścieżki jest w jednym a jej koniec w drugim, to ta krawędź w tym drzewie na pewno należy do ścieżki z $k$ do $l$. Stąd z założenia mamy, że dowolna krawędź $\left( k, l \right)$, która nie należy do drzewa ma większy koszt od $c_{ij}$ i tym samym krawędź $\left( i, j \right)$ ma najmniejszy koszt spośród wszystkich krawędzi łączących poddrzewo $S$ z $\overline{S}$, co jest warunkiem optymalnego cięcia, zaś drzewo je spełniające jest MST.
\end{itemize}
