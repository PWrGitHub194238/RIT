\chapter{Zakończenie}
\thispagestyle{chapterBeginStyle}





Celem powyższej pracy była dogłębna analiza oraz implementacja heurystycznego algorytmu \textsc{Tabu Search}, rozwiązującego problem \textsc{Recoverable Robust Incremental Minimum Spanning Tree} dla dyskretnego zbioru scenariuszy oraz zbadanie, na drodze wykonywanych eksperymentów, sposobu jego zachowania się w~obliczu różnych instancji zadanego problemu, dla eksperymentalnie dobieranych parametrów, wymaganych do jego działania.
Ze względu na klasę problemu, do której należy rozwiązywane zagadnienie (klasa problemów \textsc{NP}-trudnych, nieaproksymowalnych), zostaliśmy zmuszeni do oszacowywania jakości otrzymywanych wyników, w~sensie stopnia ich oddalenia od optymalnych rozwiązań, przez przyjęcie za optymalne rozwiązanie największych dolnych ograniczeń, jakie udało nam się otrzymywać dla zadawanych danych, przy wykorzystaniu modelu liniowego, prezentowanego w~pracy~\cite{DBLP:journals/corr/HradovichKZ16}.
Jej autorzy skupili się na bardzo podobnym problemie, jednak dotyczącym kosztów ciągłych dla krawędzi (w odróżnieniu od nas, gdzie my badaliśmy możliwości odnajdywania rozwiązań dla przypadku kosztów, definiowanych przez dyskretne zbiory scenariuszy).
Otrzymane wyniki okazują się jednak bardzo dobre, wskazując na słuszność decyzji o~wykorzystaniu algorytmu \textsc{Tabu Search}, celem rozwiązywania instancji problemu \textsc{Recoverable Robust Incremental Minimum Spanning Tree}, do którego powstania doprowadziła nas szczegółowa analiza problemu, wykonana we wszystkich rozdziałach od \ref{ch:mst} do \ref{ch:localSearch}.
Otrzymywane dzięki zastosowanemu algorytmowi rozwiązania, oddalone od optimum o~mniej niż $10$ procent (gdzie otrzymywaliśmy i~rozwiązania, których ten sam wskaźnik wynosił zaledwie dziesiąte części procenta), pozwalają stwierdzić, że wykonana przez nas praca dała oczekiwane rezultaty\footnote{
	Na podstawie eksperymentów, możemy dodatkowo stwierdzić, że otrzymywane wyniki są tym bardziej zbliżone do wartości optymalnych, im większy parametr $k$ (dla problemu \textsc{Incremental Minimum Spanning Tree}) zadamy.
}.