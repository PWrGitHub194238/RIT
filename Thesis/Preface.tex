\chapter{Wstęp}
\thispagestyle{chapterBeginStyle}

xxx

Część implementacyjna pracy magisterskiej została napisana w~języku \textsf{C++} zgodnie ze standardem \textit{\textsc{ISO/IEC 14882:2014}} z~wykorzystaniem środowiska programistycznego \textsc{Eclipse} w~wersji \textsc{$4.5.0$} (\textsc{Mars}), debuggerów \textsc{GDB} oraz \textsc{Valgrind} (w wersji \textsc{$3.10.1$}).
Tekst poniższej pracy został złożony w~systemie \LaTeX~, między innymi z wykorzystaniem podstawowych pakietów do reprezentacji kodu programistycznego: \textsc{listings} oraz \textsc{algorithm2e}.
Do składu wszelkich ilustracji użyto aplikacji internetowej \textsc{draw.io}\footnote{
	Adres strony internetowej: \url{https://www.draw.io/}.
} oraz programu \textsc{Inkscape} w~wersji \textsc{$0.91$}.
Do wygenerowania przedstawionych w~pracy wykresów dwu- i~trójwymiarowych zastosowano programy: \textsc{Octave} (w wersji \textsc{$3.8.2$}) oraz \textsc{croppdf} (do korekty otrzymanych wykresów).
Całość została skompilowana pod kontrolą systemu \textsc{Linux Ubuntu 15.04 (Vivid Vervet)}.

\begin{itemize}
	\item Trzeba zrobić wstęp o programowaniu liniowym.
	\begin{itemize}
		\item Klasa problemów 0-1 ($\mathscr{C}$): kombinatroyczne (ścieżka, drzewo), plecak, pokrycie zbioru
		\item opowiedzieć o unimodularności, relaksacji Lagrangiana, warunkach komplementarności
	\end{itemize}
	\item Wstęp o odpornej optymalizacji.
\end{itemize}