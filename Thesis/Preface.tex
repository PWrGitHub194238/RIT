\chapter{Wstęp}
\thispagestyle{chapterBeginStyle}

W poniższej pracy zajmujemy się \textsc{NP}-trudnym, w pełni nieaproksymowalnym~\cite[$6$]{Kasperski2014} problemem odpornej optymalizacji dyskretnej dla minimalnego drzewa rozpinającego w wersji \textsc{incremental} z możliwością poprawy rozwiązania dla dyskretnego zbioru scenariuszy. W związku z klasą, do której powyższy problem należy, będziemy stosowali podejście heurystyczne w celu przybliżenia odnajdywanych rozwiązań do ich optymalnych odpowiedników dla konkretnych instancji problemów. Naszym celem było zbadanie sposobu zachowania się jednego z takich algorytmów (\textsc{Tabu Search}, którego opis znajdziemy w rozdziale \ref{ch:localSearch}) dla tak zadanego problemu, co zostało wykonane w części eksperymentalnej (patrz rozdział \ref{ch:exp}). Główną ideą rozpatrywanego problemu jest odnalezienie takiego drzewa rozpinającego w grafie, które dla małej liczby zmian (w obliczu pojawienia się nowych kosztów krawędzi) nadal pozostaje rozwiązaniem najlepszym. Aby zaprezentować całość problemu, kolejno w rozdziałach \ref{ch:mst}, \ref{ch:minmax} wyjaśniamy podstawowe pojęcia dotyczące zagadnienia minimalnego drzewa rozpinającego oraz optymalizacji odpornej, aby w następnych trzech zbudować algorytmy, służące nam ostatecznie do heurystycznego wyznaczania rozwiązań, będących możliwie najbliżej maksymalnych dolnych ograniczeń dla zadanych instancji problemu. W pracy przedstawiamy zagadnienia dotyczące programowania liniowego (w rozdziale \ref{ch:linearprog}), na ich podstawie konstruujemy ideę, której implementacje przedstawiamy w następnym rozdziale, który jest poświęcony szybkiemu algorytmowi, opartemu na przeszukiwaniu binarnym, rozwiązującemu problem \textsc{Incremental Minimum Spanning Tree}. Tak skonstruowany algorytm wykorzystujemy następnie w rozdziale \ref{ch:localSearch}, gdzie skupiamy się na głównym problemie tej pracy.

Część implementacyjna pracy magisterskiej została napisana w~języku \textsf{C++} zgodnie ze standardem \textit{\textsc{ISO/IEC 14882:2014}} z~wykorzystaniem środowiska programistycznego \textsc{Eclipse} w~wersji \textsc{$4.5.1$} (\textsc{Mars}), debuggerów \textsc{GDB} oraz \textsc{Valgrind} (w wersji \textsc{$3.10.1$}).
Tekst poniższej pracy został złożony w~systemie \LaTeX~, między innymi z wykorzystaniem podstawowych pakietów do reprezentacji kodu programistycznego: \textsc{minted} oraz \textsc{algorithm2e}.
Do składu wszelkich ilustracji użyto aplikacji internetowej \textsc{draw.io}\footnote{
	Adres strony internetowej: \url{https://www.draw.io/}.
} oraz programu \textsc{Inkscape} w~wersji \textsc{$0.91$}.
Do wygenerowania przedstawionych w~pracy wykresów dwu- i~trójwymiarowych zastosowano programy: \textsc{Octave} (w wersji od \textsc{$3.8.2$} do \textsc{$4.0.0$}) oraz \textsc{croppdf} (do korekty otrzymanych wykresów).
Całość została skompilowana pod kontrolą systemu \textsc{Linux Ubuntu 15.10 (Wily Werewolf)}.